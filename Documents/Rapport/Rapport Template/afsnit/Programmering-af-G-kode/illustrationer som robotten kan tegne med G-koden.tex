\subsection{Illustrationer som robotten kan tegne med G-koden}
Ud fra G-koden beskrevet i ovenstående er det nu muligt, at lave flere forskellige slags illustrationer. 
Det er muligt at lave stort set alle logoer, som består af hvid og én anden farve. Disse kan i øvrigt laves på forskellige måder. Det er muligt at lave kanten, fyldet eller begge dele. Det kan i øvrigt justeres hvor mange pixels der tegnes med, så der kvaliteten af logoet kan reguleres. 
Foruden dette kan der også tegnes billeder med tre forskellige gråtoner, som er hvid, lys grå og mørk grå. 
Det er dog ikke alle billeder som kan tegnes på denne måde. Det kræver at billedet hverken er meget mørkt eller meget lyst, samt at der er tydelige kontraster i billedet. Hvis man eksempelvis prøver at tegne en meget mørk figur, som har en meget mørk fremtid, vil der ikke kunne skelnes mellem dette. 
 For at opnå en højere kvalitet kan man i ligesom med logoer også øge antallet af pixels. Desuden kan man med billeder også indstille grænserne for kontrasten i billedet. Dette betyder at man kan bestemme hvilken nuance der i billedet skal være hvid, lys grå eller mørk grå. 