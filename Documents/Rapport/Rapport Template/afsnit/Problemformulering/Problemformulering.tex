\section{Problemformulering}
I dette projekt skal en metode udvikles til, at konvertere billeder til en illustration, som kan tegnes af en robot. Koden, som anvendes til at tegne billedet, kaldes G-kode og laves i Mathematica. G-kode består af koordinater, linjenummer og koder, som fortæller hvornår robotten skal tegne, samt hvornår den skal spidse blyanten den tegner med. Kodningen skal virke således, at den vil kunne få robotten til at tegne flere forskellige billeder. Foruden dette skal en blyantspidser modificeres, så den ikke skal tændes manuelt, men starter automatisk når blyanten stikkes ned i den af robotten.
