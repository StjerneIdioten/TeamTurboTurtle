\section{Konklusion}
Dette projekt gik ud på at få en robot til at tegne et billede med en blyant, samt spidse blyanten når det var nødvendigt. Det er lykkedes at automatisere en blyantspidser ved at ændre på det elektriske kredsløb. Måden hvorpå dette lykkedes var ved hjælp af lysfølsom transistor overfor en infrarød LED. Kommer der lys igennem denne lysfølsomme transistor, som er koblet sammen med det inverterende input på en operationsforstærker, gør dette at det inverterende input bliver større end det ikke-inverterende input, og således lukkes der for videre strømtilførsel. I en modsat situation, hvor blyanten stikkes i, bliver det ikke-inverterende input større end det inverterende input, og der går derfor en strøm, som tænder motoren og dermed starter blyantspidseren. I Mathematica er lavet G-koder, der beskriver hvorledes robotten skal bevæge sig, og fortæller om den skal tegne eller ej, samt hvilke koordinater den skal tegne. Robotten kan både tegne logoer, samt billeder med grå nuancer. Desuden er det bestemt og angivet hvor ofte blyanten skal spidses i blyantspidseren. 
Opgaven er således opfyldt, tilfredsstillende resultater er opnået, og det er nu muligt at tegne flere forskellige slags illustrationer og spidse med den modificerede blyantspidser med en robot. 