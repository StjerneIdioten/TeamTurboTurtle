\documentclass[11pt]{article}
\usepackage[utf8]{inputenc}
\usepackage{amsmath}
\usepackage[a4paper,top=3cm,bottom=3cm,left=3cm,right=3cm]{geometry}
\usepackage{times}
\begin{document}
\setlength{\baselineskip}{1.41\baselineskip}
\parindent=10pt
\parskip=2.6mm


I denne rapport beskrives hvordan en metode er udviklet til, at konvertere billeder til en illustration, som en robot kan tegne.  Illustrationen er konstrueret af G-koder, som er kodning, lavet i Mathematica, der beskriver linjenummer, om robotten skal tegne eller lave luftkoordinater samt x, y og z-koordinater. G-koden er lavet således, at robotten er i stand til at tegne flere forskellige billeder. Da robotten tegner med en blyant er der i G-koden taget højde for slid på blyanten og længden på blyanten. Foruden dette, fortæller G-koden også hvornår blyanten skal spidses. Da blyanten bliver slidt undervejs, er det nødvendigt at spidse den når billedet tegnes. Dette gøres med en blyantspidser, som er modificeret så den automatisk starter, når blyanten sættes i den. Det beskrives i rapporten hvordan det elektriske kredsløb er lavet i blyantspidseren samt hvorledes hver komponent har indflydelse i kredsløbet. 


\end{document}