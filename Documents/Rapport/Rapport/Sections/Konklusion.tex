\section{Konklusion}

I dette projekt integrerede vi en ATmega32A microcontroller i en Scalextric slot car, således den selv kunne kortlægge en vilkårlig bane, og vælge den optimale hastighed på de forskellige baneelementer. En optisk lap sensor, et 3 akset accellerometer/gyro og en hall sensor anvendt som tachometer blev implementeret som inputs til microcontrolleren, så den kunne finde ud af, hvor på banen den var og hvor hurtigt den kørte. Derudover installerede vi en elektrisk bremse, der på signal fra ATmega'en kortsluttede motorterminalerne. \par
Oprindeligt var planen at designe en elektromagnet til at bremse bilen, men det viste sig at være meget besværligt at implementere, da den ville blive relativt tung og bruge meget strøm for at være effektiv. En kortslutningsbremse var noget simplere og krævede forholdsvis få eksterne komponenter for at få til at virke. \par
Det har vist sig at være svært at differentiere imellem inder- og yderbanesving, så vi har været nødsaget til at sørge for at lave en slags fail safe, så bilen aldrig tror den befinder sig i yderbanen, når den egentlig er i inderbanen. Det omvendte scenarie er ikke så stort et problem da bilen ikke falder af banen, hvis den kører efter den langsommere inderbanehastighed i yderbanen.\par
Hvis et lignende projekt skal laves i fremtiden, vil vi forsøge at lægge mere vægt på at få hardware-delen færdig hurtigere, idet resten af projektet hænger på at alle sensorer virker som de skal. Vi brugte også en del tid på at debugge vha. bluetooth, hvor det havde været bedre bare at designe en simplere analog løsning som eksempelvis LED'er der giver visuel feedback.