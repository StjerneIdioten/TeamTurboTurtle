\section{Indledning}

Denne rapport er udarbejdet af 6 studerende, der læser til civilingeniør i robotteknologi på SDU, det tekniske fakultet.\\

Det vil være en fordel hvis læseren af denne rapport, har et grundlæggende kendskab til programmering og elektronik. Dette skyldes at projektet hovedsagligt går ud på at benytte elektronik og programmering til at løse opgaven.\\
I dette projekt er formålet at få en scalectrix bil til at kører autonomt og forsøge at sætte den hurtigste omgangstid på en vilkårlig bane. Der skal tages højde for hastighed i sving og på langside. Bilen skal udstyres med forskellige sensorer for at kunne kortlægge banen og vide hvornår målstregen passeres. Til sidst er der foretaget en vurdering af projektet i helhed.\\
Projektet er lavet som en gruppeopgave, hvor alle arbejdsopgaver er fordelt ud blandt medlemmerne. Hver person har dog også et ansvar for at sætte sig ind i projektet som helhed og forstå de enkelte dele de ikke selv har været med til. Rapporten er inddelt således at hardware og software har afsnit for sig, med referencer frem og tilbage til de enkelte tilfælde, hvor de to ting hører sammen. Den software der styrer bilens "hjerne" har fået et afsnit for sig selv da dette er en længere udredning. Rapporten kan med fordeles læses på pc da links er direkte integreret i pdf filen også til referencer i selve dokumentet.


\subsection{Problemformulering}

Vi ønsker at ombygge en scalextric bil til at kunne kører autonomt på en ukendt bane. Den skal selv formå at opmåle banen og derudfra regulere sine egne parametre, med det formål at opnå den hurtigst mulige omgangstid.
\begin{itemize}
	\item Hvordan får vi bilen til at bremse ?
	\item Hvordan får vi bilen tll at holden en jævn hastighed ?
	\item Hvordan får vi mappet banen op ?
	\item Hvordan får bilen til at reagere på det givne map ?
\end{itemize}

\subsection{Kravspecifikation}

Vi har fået stillet disse følgende oblikatoriske krav til projektet

\begin{itemize}
	\item Den udleverede bil skal benyttes.
	\item Der skal anvendes en ATmega32 microcontroller i forbindelse med projektet.
	\item Kommunikationsprotokollen skal overholdes af hensyn til test (se nærmere i den efterfølgende gennemgang) – Det er dog tilladt at lave tilføjelser til protokollen. Det udleverede trådløse Bluetooth-modul skal benyttes.
	\item Der skal gøres et reelt forsøg på at gennemføre banen på den bedste tid – det er med andre ord ikke nok at lave et projekt, hvor bilen kører banen rundt med konstant fart.
	\item Kommunikation mellem PC og bil kan foregå via et terminalprogram, eller man er velkommen til egenhændigt at udvikle et PC-baseret program (C++, C\#, Java, etc.) til formålet.
	\item Der skal anvendes en elektromagnetisk sensor og/eller aktuator i projektet. Der må ikke benyttes permanente magneter.
\end{itemize}


\subsection{Projektafgrænsning}

Vi har afgrænset os til at programmere i asmebly, udover dette har vi begrænset os til at bruge vores eget accelometer med gyro (MPU5050).
Vi vil fremstille en bil der skal kunne køre udlukkende af sig selv, ved hjælp af den kode der bliver skrevet til den.
Som en del af vores projekt har vi stillet nogle krav som vi vil opfylde.

\begin{itemize}
	\item Bilen skal kunne køre autonomt
	\item Bilen skal kunne forsøge at sætte den hurtigtiste omgangstid
	\item Bilen skal kunne opretholde sin funktion ved kortvarige udfald på strømforsyningen(Blackout og Brownout)
	\item Der skal laves et monitoreringssuite til pc der kan overvåge bilens performance
	\item Bilen skal kunne bremse
	\item Vi vil kunne se forskel på store og små sving
\end{itemize}

Vi har tænkt os lave nogle forskellig test på bilen for at kunne optimere på dens performance. En af de test vi har tænkt os at lave er se på dens bremselængde ved at lave en kvantitativ test.

De teori områder vi kommmer indpå har vi tænkt os at dele op i to områder Hardware og software

\item[Hardware] \hfill
		\begin{itemize}
			\item Accllerometer
			\item Lap sensor
			\item Tachometer
			\item Bremse
			\item elektromagnet
		\end{itemize}
		
\item[Software] \hfill
		\begin{itemize}
			\item Strukture af kode
			\item Interfacing af hardware
			\item Kommunikations protokol
			\item GUI (Graphical user interface)
		\end{itemize}
 

\subsection{Tidsplan}