\section{Indledning}

Denne rapport er udarbejdet af 6 studerende, der læser til civilingeniør i robotteknologi på SDU, det tekniske fakultet.\\

Det vil være en fordel hvis læseren af denne rapport har en grundlæggende kendskab til programmering og elektronik. Dette skyldes at projektet hovedsageligt går ud på at benytte elektronik og programmering til at løse opgaven.\\

I dette projekt er formålet at finde en metode til at opnå den hurtigste omgangstid på en vilkårlig bane. Der vil tages højde for hastighed i sving og på langside. Bilen udstyres med forskellige sensorer for at kunne kortlægge banen og vide hvornår målstregen passeres. Til sidst er der foretaget en vurdering af projektet i helhed.\\

Projektet er lavet i en gruppe, hvor arbejdsopgaver fordeles. Hvert medlem har dog ansvar for at sætte sig ind i dele, de ikke selv har været en del af. Gruppen har haft møde en mindst gang ugentligt, hvor der blev fremlagt fremskridt, diskussion af problemer undervejs samt uddelegering af nye arbejdsopgaver. Gruppen har i forløbet haft flere møder med vejleder, for at diskutere retning og eventuelle spørgsmål er blevet afklaret.\\

Rapporten er opbygget kronologisk og de forskellige kapitler, der har relevans for hinanden, vil ligge samlet. Således er der en rødtråd igennem rapporten.\\

\subsection{Kravspecifikation}

\subsection{Problemformulering}

\subsection{Projektafgrænsning}

\subsection{Tidsplan}