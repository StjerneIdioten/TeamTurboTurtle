\section{AI}


\subsection{Indledning}
Med de sensorer vi benytter i det her projekt, er det besluttet at den databehandling, som microcontrolleren laver fungerer bedst som interrupt funktioner. På denne måde benytter vi, at data bliver opdateret i interrupt funktioner, og derfor bliver behandlet med det samme.
\\

Det vigtigste interrupt er Hall-interruptet, se figur \ref{fig:Hall Flowchart}. Denne sker hver gang, hjulet er kørt en tolvtedel rundt, og giver derfor en afstand. Da der måles hvor lang tid, der går mellem to af disse, fås derfor også en periode, som benyttes som en psuodo hastighed. Det er i dette interrupt at det meste af koden ligger.
\\

Det andet interrupt er Lap-interruptet, se figur \ref{fig:Lap Flowchart}. Denne skifter mellem bane omgangene, og laver de beregninger, der skal, ske når vi går fra mapping runden til den første runde, samt resetter det map vi laver.
\\

Koden laver dermed intet  i tidsrummet mellem interrupts, hvilket gør det muligt at lave Bluetooth kommunikation.

\subsection{Runder}

Da bilen først skal mappe banen op, og derefter køre efter dette map, er der tre runde-tilstande:

\begin{itemize}
\item Preround: Bilen er sat ned på banen, og venter på at ramme målstregen.
\item Mapping round: Bilen er ved at sætte et kort, eller et ``map'' op af banen.
\item Run time: Bilen er ved at køre efter det målte map.
\end{itemize}

\subsubsection{Preround}
I preround kører bilen blot med jævn hastighed, indtil den rammer målstregen. Der er ikke nogen grund til, at den skulle gøre noget, før dette sker, og det eneste relevante interrupt er lap sensoren, som i dette tilfælde blot tæller lap counteren op til 1.

\subsubsection{Mapping round}
I mapping round laver bilen et map af banen. For at mappe banen op, og benytte dette map til at optimerer bilens omgangstid, benyttes to sensorer. Den ene giver vinkelhastigheden rundt om Z-aksen, hvilket benyttes til at bestemme, om bilen befinder sig på et lige stykke bane, eller i et sving. Den anden er en hall sensor, som laver et interrupt, når den måler en vis værdi, hvilket sker tolv gange på en hjulomdrejning, hvilket giver en afstand samt en ``hastighed''. 
\\\

Ved så at benytte Hall-funktionen, se afsnit \ref{Hall-funktion}, bestemmer vi hvor lang hver banestykke er, før der kommer et baneskift, som vi undersøger med Skift-funktionen, se afsnit \ref{Skift funktion}
\\
Derefter benytter vi hastighedskontrollen, som bliver beskrevet i afsnit \ref{sec:Hastighedskontrol}.
\\
Når mapping runden er færdig skal Lap-interruptet gemme de nuværende værdier af afstanden og banetypen i rammene. 

\subsubsection{Run time}

Når bilen har lavet et map af banen, skal den kører efter det. Dette gøres ved i starten af runden, med Lap-interruptet, at indlæse bane type og længde fra toppen af listen. Når der sker et Hall-interrupt, dekrementerer vi den nuværende del af bane længen. Når denne går i nul indlæses de næste emner fra listen, banetype og længde. På denne måde ved vi altid, hvor på banen vi befinder os.
\\
Ved så at branche ud alt efter vores nuværende banetype kan vi så sætte den ønskede ``hastighed'', som så gives videre til hastigheds-kontrol funktionen.
\\
Der er to special-tilfælde. På vej ud af et sving sættes ``hastigheden'' til det samme som på et lige banestykke, og på vej ind i et sving bremses der en fast afstand fra et sving, bestemt fra hvor lang tid der højst skal bremses fra fuld hastighed til den hastighed vi kører med i et sving.
