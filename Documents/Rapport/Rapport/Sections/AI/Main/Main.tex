\section{AI}
\label{sec:AI}

\subsection{Indledning}
Med de sensorer vi benytter i det her projekt, er det besluttet at den databehandling som microcontrolleren laver fungerer bedst som interrupt funktioner. På denne måde benytter vi at data bliver opdateret i interrupt funktioner, og derfor bliver behandlet med det samme.
\\

Det vigtigste interrupt er Hall-interruptet, se figur \ref{fig:Hall Flowchart}. Denne sker hver gang hjulet er kørt en tolvtedel rundt, og giver derfor en afstand. Da der måles hvor lang tid der går mellem disse to fås derfor også en periode, som så benyttes som en psuodo hastighed. Det er i dette interrupt at det meste af koden ligger.
\\

Det andet interrupt er Lap-interruptet, se figur \ref{fig:Lap Flowchart}. Denne skifter mellem bane omgangene, og laver de beregninger der skal ske når vi går fra mapping runden til den første runde, samt resetter det map vi laver.
\\

Koden laver dermed intet  i tidsrummet mellem interrupts, hvilket gør det muligt at lave Bluetooth kommunikation.

\subsection{Runder}

Da bilen først skal mappe banen op og derefter køre efter dette map, er der tre runde-tilstande:

\begin{itemize}
\item Preround: Bilen er sat ned på banen og venter på at ramme målstregen.
\item Mapping: Bilen er ved at sætte et kort, eller et ``map'' op af banen.
\item Run time: Bilen er ved at køre efter det målte map.
\end{itemize}

\subsubsection{Preround}
I preround kører bilen blot med jævn hastighed indtil den rammer målstregen. Der er ikke nogen grund til at den skulle gøre noget før dette sker, og det eneste relevante interrupt er lap sensoren, som i dette tilfælde blot tæller lap counteren op til 1.

\subsubsection{Mapping round}
I mapping round laver bilen et map af banen. For at mappe banen op, og benytte dette map til at optimerer bilens omgangstid, benyttes to sensorer. Den ene giver vinkelhastigheden rundt om Z-aksen, hvilket benyttes til at bestemme om bilen befinder sig på et lige stykke bane, eller i et sving. Den anden er en hall sensor, som laver et interrupt når den måler en vis værdi, hvilket sker tolv gange på en hjulomdrejning, hvilket giver en afstand samt en ``hastighed''.
\\\


Når der kommer et Hall-interrupt i mapping serien starter vi med at se hvilken banetype gyroen siger at vi befinder os på, se figur figur \ref{fig:Skift Flowchart}. Vi sammenligner så denne med den banetype vi befandt os på i det forrige Hall-interrupt. Hvis disse matcher, eller hvis denne kun viser en forskel på et lille eller et stort sving, så inkrementerer vi længden af det nuværende banestykke med én. Hvis ikke, så gemmer vi længden og typen af banestykket i rammene, skifter banestykke, og sætter længden til 1. På denne måde får vi så længden og typen af alle banestykker i rækkefølge, hvilket nemt kan bruges til at følge banen i de øvrige runder.
\\
Derefter benytter vi hastighedskontrollen, som bliver beskrevet her: HENVIS TIL AFSNIT.
\\\
Når mapping runden er færdig skal Lap-interruptet gemme de nuværende værdier af afstanden og banetypen i rammene. 

\subsubsection{Run time}

Når bilen har lavet et map af banen skal den kører efter det. Dette gøres ved i starten af runden, med Lap-interruptet, at indlæse bane type og længde fra toppen af listen. Når der sker et Hall-interrupt dekrementerer vi den nuværende del af bane længen. Når denne går i nul indlæses de næste emner fra listen, banetype og længde. På denne måde ved vi altid hvor på banen vi befinder os.
\\\
Ved så at branche ud alt efter vores nuværende banetype kan vi så sætte den ønskede ``hastighed'', som så gives videre til hastigheds-kontrol funktionen.
\\
Der er to special tilfælde. På vej ud af et sving sættes ``hastigheden'' til det samme som på et lige banestykke, og på vej ind i et sving bremses der INDSÆT HVAD DER SKER HER NÅR VI ER FÆRDIGE MED DET.